\section{Conclusion}
\label{sec:conclusion}


In this paper, we design a mechanism based on group mobility and multi-criteria named Grouped NetSel-RF. Thus, we create an algorithm that groups and manages the devices belonging to the user. Besides, we implemented NetSel-RF in group mobility management to select the network based on multi-criteria (network, user preferences, end-devices, and applications). We evaluate our mechanism in the Mininet-WiFi emulator based on the metrics of the number of handovers, the quantity of the message, and network selection time. The results show the Grouped AHP-TOPSIS and Grouped NetSel-RF mechanisms reduced the number of messages by 66.67\% concerning the traditional mechanism. Although the Grouped AHP-TOPSIS and Grouped NetSel-RF mechanisms have similar behavior, our mechanism is proactive; therefore, NetSelf-RF makes a hand-off to an AP with better conditions before the mobile device loses connection, in contrast to grouped AHP-TOPSIS.  Besides, Grouped NetSel-RF presented the lowest rate of 802.11 management frames. Indeed, our mechanism decreased by 40\% the authentication frame, 46.15\% the de-authentication frame, and 40\% the association request and association response frames. Finally, Grouped NetSel-RF takes 2.71 ms more compare to SSF selecting the network, since process more criteria to take into account the needs of the group and group mobility increases handover time.



As future work, we intend to implement and analyze the performance of grouped NetSel-RF in heterogeneous networks.

