\section{Related Work}
\label{sec:relativework}

\subsection{Resource Allocation}

In \cite{209-27}, the authors designed an HM scheme based on BW reservation policies sensitive to the traffic class. This scheme reserves the BW in each AP according to service classes (e.g., Best Effort, Background, Video, and Voice) and handover processes. However, if the BW is unused, it must remain available. In this way, the handover traffic always has the necessary BW and achieves handovers without data loss with a minimum delay of 60 ms. This scheme has the following limitations: i) low scalability of the network due to the disuse of network resources, and ii) high handover delay for LLC requirements, because of the insufficient traffic classes, the lack of efficient resource management.

In \cite{125}, a dynamic QoS based IP HM procedure was proposed to handle the applicati-on-centric mobility management. Such QoS based handover process ensures the required quality level for the on-going connections according to the policies enforced by the SDN controller. This HM procedure uses SDN to identify (before handover) the appropriate route to provide the required bit rate according to the applications QoS. At the same time, SDN allows the IP address to be maintained to avoid disconnection during the handover. In this manner, this proposal omits the association phase and reduces the delay up to 50 ms. However, the handover delay is still excessive for LLC requirements which needs handover delay less than 10 ms.

In \cite{127}, the authors presented a proactive approach to radio channel assignment in conjunction with HM. This approach determines the channel queue and channel occupancy time in AP. For when a handover occurs, the approach selects the objective AP with the lowest channel occupancy. Thus, this approach avoids handover blocking. Although it can generate critical delays for LLC-type applications as a result of receiving traffic flows without differentiating whether it is delay-sensitive or not.

In \cite{a131}, the authors proposed a Machine Learning based method to find an optimal handover mechanism. This method allows us to predict whether the handover that is going to happen will maintain the throughput, optimizing resource allocation between APs. However, to maintain an algorithm with predictive levels of acts, at first, we need data to train this algorithm. Therefore, there will be quite an amount of wrong decisions about handover prediction causing QoS degradation.

\subsection{Proactive Service Replication}
In \cite{205}, the researchers propose the proactive replication of stateless application instances in neighbouring AP. This proposal maintains and updates instances in the neighbouring AP with application and connection data. When the handover occurs, the instance must update less amount of data (\textgreater 10 MB). Although devices perform handovers without data loss, there is a downtime of more than 500 ms. Therefore, this proposal violates the LLC requirements.

In \cite{201}, the authors optimized the proactive copying of application connection information through dockers in the neighbouring APs. This proposal uses mobility prediction algorithms to minimize containers in neighboring APs. This way, this work achieves an excellent rate of 97.5\% for seamless handover (with at least 4 APs). However, this proposal ignores the evaluation of handover delay. At the same time, the high waste of network resources used in dockers, makes the present solution inefficient to meet the rigorous LLC requirements.

In \cite{217}, researchers introduced the full-state application migration mechanism based on a predefined path. The mechanism uses Checkpoint-Restore in User Space (CRIU) to save the executing application state in a container before handover. Subsequently, the container is copied to the destination AP, and the application is restarted according to the CRIU checkpoint. Although this mechanism conserves all application data, it has a downtime (\textgreater 1000ms) that impairs the continuity of applications with LLC requirements.

In \cite{903}, the authors proposed a vehicular MEC architecture instead of simply offloading LTE infrastructure. Routing all the packets with the MEC network achieves Vehicle to Infrastructure (V2I) communications with very low packet delay (10 - 30ms). Also, this architecture provides seamless handover with Distributed Mobility Management (DMM) in the MEC network. Nevertheless, in order to achieve seamless handovers with low delay, this architecture makes use of requests on servers close to the user. When the servers for the services are far away from the vehicle, any request outside the MEC network will have adverse effects on the seamless handover and delay times. 

\subsection{Network Virtualization}

In \cite{101}, the authors propose BYON to create network slices with dedicated resources according to a set of QoS requirements. BYON has an SDN controller to configure each slice in an additional AP interface. Furthermore, the SDN controller enables APs to store flows to avoid packet loss during handover. BYON achieves handovers without packet loss in less than 65 ms. However, BYON has high handover delay that degrades LLC requirements. Furthermore, it is few scalable, given the difficulty of adding the necessary interfaces in each AP.

In \cite{209}, the authors propose ADE2WiNFV to provide NaaS, i.e., to offer custom network slices according to a set of QoS requirements. ADE2WiNFV combines SDN and NFV to virtualize/assign APs, network resources and NFs, and thus offer independent network slices. Additionally, ADE2WiNFV implements Protocol-Oblivious Forwarding (POF) to route flows to their corresponding VAP, even when the handover occurs. In this manner, ADE2WiNFV meets the applications QoS with a minimum handover delay of 220 ms. However, ADE2WiNFV has the following disadvantages: i) excessive handover delay compared with LLC requirements, ii) lack of resource reallocation in physical APs (given the handovers of the devices), and iii) high downtime (\textgreater 500 ms). To sump up, ADE2WiNFV degrades the Qos requirements as LLC.

In \cite{26}, the authors present Odin to introduce the concept of Light Virtual AP (LVAP) based on SDN. LVAP gives the illusion that each device has its own AP. For when the handover occurs, the SDN controller only has to change (in LVAP) the registry of the linked AP. Thus, the devices skip the authentication phase and reduce the handover delay up to 1 ms. However, a more significan number of devices considerably increases the handover delay due to rising control traffic. Therefore, although Odin achieves delays according to the LLC requirements, it needs to improve its handover mechanism.

In \cite{103}, the researchers proposed an open enterprise WiFi solution based on virtual APs, managed by a central WLAN controller. It allows seamless handovers between APs in different channels, maintaining the QoS of real-time services. This is achieved by omitting the discovery and authentication phases of the handover. The scheme assigns each device a VAP and each AP an additional interface. Furthermore through an SDN controller, the virtual APs employ the additional AP interface to discover the available channel (for handover) in the neighbouring APs, and authenticate the device with the discovered channel. In this way, The device thinks it's still in the same AP, reducing the handover delay up to 22 ms. However, this virtual APs scheme degrades compliance with the LLC requirements due to the high handover delay.\\

In \cite{327}, researchers propose a new architecture for LTE and WiFi networks to achieve low latency. This solution use SDN and NFV to create LVAPs. In order to meet low latency, they use a Packet Data Network Gateway (P-GW)\footnote{P-GW allows traffic mapping from LTE to WiFi, and a Wireless Access Gateway interacts with the user as an LVAP} that serves to download and extract the data to a Wireless Access Gateway. The LVAP decreases the handover latency using the same BSSID with all the LVAPs, making the device think it remains in the same network. However, the network ignores the available resources in the destination AP when it triggers a handover. This may generate latency in case there are many users or few resources in the destination LVAP. \\



%---------------------------------------------------------------------------------------------------


