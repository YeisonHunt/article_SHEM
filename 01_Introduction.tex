\section{Introduction}
\label{sec:introduction}

Fifth-Generation of mobile communication networks, also called 5G, supports emerging application requirements that demand seamless handovers to satisfy Low-Latency Communication (LLC) requirement \cite{227}. These networks deploy numerous Access Points (AP), also called gNB\footnote{The AP in 5G is known as gNB or gNodeB, meaning for Next-Generation NodeB and is a successor term to eNB, pertaining to 4G networks \cite{590-v2}}, to improve network resource utilization and enhance the Quality of Service (QoS) expected by mobile users \cite{227,104}. The process that handles the (dis)connection of a device when it moves between APs is named Handover Management (HM). However, device mobility and network density generates long handover delays (e.g., delays higher than 150 ms \cite{403}) that degrades the network performance and diminish QoS \cite{216}. These handover delays constitutes a limitation for LLC use cases as Augmented Reality (AR), where latency requirement is of 10 ms since the long lag between images can cause user disorientation \cite{203,326,239,328}. Therefore, 5G needs to optimize HM aiming to meet LLC requirement and guarantee application connectivity.

%3GPP TS 38.300 Sept. 2018. TSG RAN; NR; Overall Description; Stage 2 (Release 15), v15.3.0. 3GPP TS 38.300.

%WiFi corresponde al estándar IEEE 802.11 que comprende un conjunto de normas para controlar el acceso al medio y las características físicas de las redes de área local inalámbricas (WLANS). En consecuencia, el WiFi ha sido considerado como una tecnología de acceso inalámbrico.

In 5G, WiFi\footnote{WiFi corresponds to the 802.11 standard developed by the Institute of Electrical and Electronics Engineers (IEEE). This standard comprises a set of standards for controlling access to the medium and physical characteristics of Wireless Local Area Networks (WLANS) \cite{607}. Consequently, WiFi has been considered as a Radio Access Technology (RAT).} will play a key role, since it represents a more affordable, faster, and reliable communication alternative\footnote{Furthermore, it is estimated that wifi and mobile traffic will represent 63\% of all IP traffic by 2021 \cite{240}.} to other wireless technologies such as WiMax and Satellite. Meanwhile, WiFi addresses compliance with LLC requirement in HM through three standards: Service Differentiation (DiffServ - 802.11e standard), Radio Resource Management (RRM - 802.11k standard), and Fast Transition (FT - 802.11r). DiffServ classifies network traffic into four service classes (Background, Best Effort, Video, and Voice) to give different access times to the medium, without reducing the handover delay \cite{329,31}. RRM simplifies the proactive search for the destination AP by creating a list of available channels from neighboring APs \cite{113}. This way, RRM reduces the handover delay in the discovery phase up to 120 ms \cite{403}. FT allows the AP to store the encryption keys of all network APs \cite{16}. Thus, the devices diminish the authentication delay and achieve a minimum handover delay of 50 ms. Although WiFi still lacks mechanisms that optimize HM to improve network resources management and meet LLC requirement.


Recent research in HM proposes mechanisms such as Resource Allocation, Proactive Service Replication, and Network Virtualization to meet LLC requirement. Resource Allocation \cite{209-27, 125, 127, a131} operates by the reservation of available network resources based on competing for application demands (e.g., link BW\footnote{Bandwidth} and buffer space in APs). Nevertheless, all demands are impossible to meet, since some applications may receive fewer network resources, increasing the latency in LLC. Proactive Service Replication \cite{205,201,217,903} operates by application instances\footnote{Virtual machines (VMs) or Dockers (\textgreater 10 MB) that can store both application and connection information \cite{205}.} deployment in nearby APs before handover using Mobile Edge Computing (MEC) cloud capabilities (processing and storage). Nonetheless, application instances must be continuously updated, resulting in inadequate use of network resources (besides the storage occupied by the instances) and hence the degradation of overall network performance \cite{205,201}. Network Virtualization \cite{101,209,26,103,327} works by custom network slices creation with Virtual AP (VAP), where each slice has dedicated resources according to the QoS requirements of one or more applications. However, the slices creation requires the resource reallocation in each AP, generating in a high downtime\footnote{Disconnection time caused by the delay of resource reallocation.} (\textgreater 500 ms) that fail LLC requirement. In conclusion, the previous mechanisms by HM evidence the difficulty of meeting LLC requirement in 5G.

%ref 251
El artículo está estructurado de la siguiente manera. En la sección 2, se describen los trabajos relacionados con nuestro problema y se explican nuestras contribuciones. La sección 3 presenta los cortes de referencia 5G e introduce los cortes V2X. Presentamos la arquitectura de slicing 5G adoptada en la Sec. 4 y dedicamos las Secciones. 5 y 6 a dar una explicación detallada sobre nuestra solución. La sección 7 está concebida para evaluar nuestros algoritmos propuestos. Por último, concluimos nuestro artículo en la sección 8.

%Ref 239
El resto del manuscrito está organizado de la siguiente manera. En la Sección 2, se proporciona material de referencia para nuestro trabajo, mientras se exploran las especificaciones 3GPP C-V2X y 5G relacionadas y la literatura sobre la división de redes para V2X. La Sección 3 proporciona una descripción general del marco de corte de red propuesto. En la Sección 4 se proporcionan detalles sobre el diseño de una porción V2X para la conducción cooperativa además del marco concebido. La Sección 5 informa la descripción de los procedimientos de gestión de la movilidad en el marco propuesto. Un caso de uso de referencia se muestra prácticamente en la Sección 6, donde se discuten los resultados obtenidos. Finalmente, la Sección 7 resume las observaciones finales y los argumentos sobre trabajos futuros.
