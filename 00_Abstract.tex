\abstract{One of the main objectives of Fifth Generation (5G) mobile communication networks is to support low latency applications up to 1ms E2E. To enable low latency communications (LLC), 5G has adopted solutions such as Network Slicing (NS) along with access point densification (also called gNB). NS allows the creation of customized logical networks (slices) according to the Quality of Service (QoS) requirements of one or more applications. On the other hand, gNB densification increases network coverage and capacity. Although these two solutions allow better resource management, it generates frequent gNB changes (Handover Management - HM) due to the mobility of the user equipment (UE). Consequently, this HM in the 5G network has two difficulties enabling LLC. First is the uncertainty in meeting the QoS requirements of the application in the target gNB, given the unknown availability of resources in the gNBs and slices. Second, the interruption of up to 3900 ms in UE communication, given the HM process. Therefore, the 5G network alone is deficient in performing the HM and meeting the LLC requirement. For this reason, we introduce SHEM to proactively select the target gNB, taking into account the available resources as the LLC requirement of the application. The evaluation results show that SHEM reduces the HM latency by approximately 3700 ms and achieves 73.5\% effectiveness in meeting the LLC requirement of the application.
}
% Keywords
\keyword{5G; Handover; Network Slicing; LCC; SHEM.}


